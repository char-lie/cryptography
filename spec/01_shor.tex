\def\ZN{\mathbb{Z}_{N}}
\chapter{Алгоритм Шора для факторизации}
\section{Введение}

Есть число $N$
\begin{equation*}
  N = a \cdot b,\qquad a \neq b
\end{equation*}

\textbf{Задача факторизации}: для числа
$N = p_{1}^{\alpha_{1}} \cdots p_{t}^{\alpha_{t}}$
\begin{itemize}
  \item либо получить каноническую форму
  \item либо получить нетривиальный делитель
\end{itemize}

Число $a \in \mathbb{Z}_{N}$ \textbf{принадлежит показателю} $\delta$, если
$\delta$ --- минимальное число, для которого
$a^{\delta} \equiv 1 \mod n$.
Обозначаем $\delta_{N}\left( a \right)$.

Показатель существует только если $a$ и $N$ взаимно просты ($a \in \ZN$)
и по сути является порядком $a$ в группе
$\ZN: \delta_{n}\left( a \right) = ord_{N}\left( a \right)$.
Если $gcd\left( a, N \right) \neq 1$, то мы нашли нетривиальный делитель.

Допустим, что для заданного $a \in \ZN$ у нас есть оракул $O_{f}$,
который возвращает показатель числа.
Если этот оракул полиномиален, то можно факторизовать число (вероятностно)

Пусть $r$ --- чётное и $gcd\left( a^{r/2} + 1, N \right) = 1$,
где $O_{f}\left( a \right) = r$
\begin{equation*}
  \begin{split}
    a^{r} \ge 1 \mod N \\
    \left( a^{r/2} - 1 \right) \cdot \left( a^{r/2} + 1 \right) \equiv
      0 \mod N
  \end{split}
\end{equation*}
Но $a^{r/2} \neq 1 \mod N$, потому что иначе $r$ --- не минимальное число
(нарушение определения показателя).
Значит, $\left( a^{r/2} - 1 \right)$ и $N$ имеет нетривиальный общий делитель.

Важно: условие $gcd\left( a^{r/2} + 1, N \right) = 1$ можно заменить на
условие $a^{r/2} + 1 \vdots N$.

\begin{affirmation}
  Пусть $N = p_{1}^{\alpha_{1}} \cdots p_{t}^{\alpha_{t}}$ и
  \begin{equation*}
    S = \left\{ a \in \ZN: ord\left( a \right) mod 2 = 0
    \lor a^{ord\left( a \right)/2} + 1 \vdots N \right\}
  \end{equation*}
  Тогда
  \begin{equation*}
    \left| S \right| \le \frac{\varphi\left( N \right)}{2^k}
  \end{equation*}
\end{affirmation}

То есть, подходящих нам $a$ очень мало.

\section{Алгоритм факторизации с оракулом}

\begin{enumerate}
  \item Генерируем такое $a$, чтобы при $r = O_{f}\left( a \right)$ выполняось
    \begin{equation*}
      r = 2 \cdot r_1, a^{r_{\alpha}} + 1 \vdots N
    \end{equation*}
  \item $gcd\left( a^{r_1} - 1, N \right)$ --- нетривиальный делитель.
\end{enumerate}

\begin{affirmation}
  В классической модели найти оракул $O_f$ не получилось.
  Шору удалось построить его в квантовой модели.
\end{affirmation}

Рассмотрим функцию $f: \ZN \rightarrow \ZN$
\begin{equation*}
  f\left( x \right) = a^{x} \mod N
\end{equation*}
Это почти непериодическая функция --- её период не кратен $N$.

\section{Квантовая система}

У нас есть $\left| 0 \right>$, $\left| 1 \right>$; кубит находится в
суперпозиции состояний
$\lambda_1 \cdot \left| 0 \right> + \lambda_2 \left| 1 \right>$, над которыми
можно выполнять унитарные операции.

Набор кубитов --- одна из возможных интерпретаций квантово-механической системы.
Другой вариант --- $N$-уровневая система, в которое есть $N$ состояний
$\left| 0 \right>$, $\left| 1 \right>$, $\dots$, $\left| N-1 \right>$, и
эти состояния ортонормированы (то есть, при измерении выпадают только эти
состояния и ничего среднего между ними).
