\chapter{Моделі джерел відкритого тексту. Ентропія на символ джерела}

При шифруванні текст перетворюється таким чином, щоб зробити його зміст
незрозумілим для того, хто не знає секретного ключа. Для побудови математичної
теорії криптографічних систем шифрування потрібно насамперед дати математичний
опис (або математичну модель) тексту та перетворень, які відбуваються з ним під
час шифрування.

\begin{definition}[Алфавіт]
    Надалі вважатимемо, що алфавіт є скінченним.
    Позначимо алфавіт як $Z_m = \left\{z_1, \dots , z_m \right\}$, де
    $z_1, \dots , z_{m}$ --- букви (символи) алфавіту.
    Елементами алфавіту можуть бути власне букви; букви та цифри;
    букви, цифри та знаки пунктуації, взагалі скінченний набір
    будь-яких символів, наприклад, танцюючі чоловічки.
    Як правило, ми будемо розглядати український,
    російський чи латинський алфавіт (малі букви) зі знаком пропуску,
    що вважається буквою, або без нього, або ж двійковий алфавіт,
    що складається з двох символів: $0$ та $1$. 
\end{definition}
\begin{definition}[Текст]
    Під текстом будемо розуміти послідовність букв деякого алфавіту.
\end{definition}

\begin{definition}[Відкритий текст] Відкритий текст (ВТ) --- це текст,
    що підлягає шифруванню.
\end{definition}
\begin{definition}[Шифрований текст]Шифрований текст (ШТ) --- це текст,
    що утворюється в результаті шифрування.
\end{definition}

Відкритий та шифрований тексти можуть бути записані як у одному й тому ж,
так і у різних алфавітах (більш докладно поняття ВТ та ШТ
розглядаються у лекції 3).

\begin{definition}[n-грама]
n-грамою називається послідовність  $n$ символів тексту, що
стоять підряд. При  ${n}$=2 це біграма, при  ${n}$=3 --- триграма.
\end{definition}

Будь-який текст має певну статистичну структуру. Для опису цієї структури
використовуються різноманітні ймовірнісні моделі мови. 

\begin{definition}[Джерело відкритого тексту]
    Джерело відкритого тексту генерує послідовність символів алфавіту 
    $x_1, x_2, \dots, x_n, \dots$ випадковим чином.
    Джерело визначається алфавітом та ймовірностями появи $n$-грам: 
    $\probability{x_{i+1}=z_1, x_{i+2}=z_2, \dots, x_{i+n}=z_n}$
    для будь-яких цілих  $n \ge 1$, $i \geq 0$
    (тут $x_1, x_2, \dots, x_n$ --- випадкові величини,
    а $z_1, \dots, z_n$ --- букви алфавіту), які мають задовольняти умовам: 

    \begin{enumerate}
        \item Вихід джерела ВТ є випадковим процесом
            з дискретним часом та множиною станів  $Z_m$
        $$\sum_{z_1, z_2, \dots, z_n \in Z_m}
            \probability{x_{i+1}=z_1, \dots, x_{i+n}=z_n}=1$$
        \item Умова узгодженості
            скінченновимірних розподілів виходу джерела ВТ:
            Для будь-якого цілого  ${s\ge 1}$
        \begin{align*}
            \sum_{z_1, z_2, \dots, z_n \in Z_m}
                \probability{x_{i+1}=z_1, \dots, x_{i+n}=z_n, \dots, x_{i+n+s}
                    =z_{n+s}} = \\
                = \probability{x_{i+1}=z_1, \dots, x_{i+n}=z_n}
        \end{align*}
    \end{enumerate}
\end{definition}

\begin{definition}[Стаціонарне джерело відкритого тексту]
    Джерело називають стаціонарним, якщо для будь-яких цілих  $n\ge 1,
    1\le i_1<\dots<i_n,j\ge 0$
    і будь-якого набору букв алфавіту  $z_1, \dots,z_n$ виконується  рівність:
    $$\probability{x_{i_1+j}=z_1, x_{i_2+j}=z_2, \dots, x_{i_n+j}=z_n}
        = \probability{x_{i_1}=z_1, x_{i_2}=z_2, \dots, x_{i_n}=z_n}$$
\end{definition}

У подальшому будемо розглядати лише стаціонарні джерела, тобто такі, у яких
немає залежності від зсуву $j$.
Для стаціонарних джерел достатньо задати ймовірності
$\probability{x_1=z_1,\dots,x_n=z_n}$ для ${n\ge 1}$.

В залежності від властивостей сумісних розподілів 
$\probability{x_1=z_1, \dots, x_n=z_n}$, $n\ge 1$
можна побудувати різні моделі джерела ВТ.

Найбільш уживаними є описані нижче чотири моделі, з яких кожна наступна все
більш адекватно відображає структуру мови. Перша з них є простою й менш за всі
враховує реальні статистичні властивості мови. Назвемо її моделлю М0.
\begin{definition}[Модель M0]
    У цій моделі джерело у кожен момент часу генерує символи
    із  $Z_m$ незалежно та рівноімовірно:
 ${P\left(x_{{i}}=z\right)=\frac{1}{m},i=1,2,\dots}$.,  
${z\in ?}$ ${Z_{{m}}}$.
\end{definition}

